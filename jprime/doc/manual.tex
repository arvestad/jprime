\documentclass[]{report}   % list options between brackets
\usepackage{}              % list packages between braces

% type user-defined commands here

\begin{document}

\title{JPrIME Manual}   % type title between braces
\author{CompBio group at SciLifeLab Stockholm}         % type author(s) between braces
\date{}    % type date between braces
\maketitle

%\begin{abstract}
%\end{abstract}

\chapter{Installation}
\section{Introduction}
JPrIME is a software package aimed at computational biology tasks, currently with an emphasis on phylogenetics.
The majority of the code is written in Java. It has been conceived and implemented by people in Jens Lagergren's, Lars
Arvestad's and Bengt Sennblad's groups at SciLifeLab Stockholm. JPrIME is available under GNU GPLv2 and GPLv3.
\section{Prerequisites}
\subsection{Java}
Java version SE 6 is required. This is subject to change to Java SE 7. Many applications rely on external packages -- see the individual
applications for more info.
\\
As an example, suppose we have installed JPrIME on a Linux machine in \texttt{$\sim$/JPrIME/1.0/}, with the main jar file \texttt{jprime.jar} residing directly
under this folder and the external libraries residing under the subfolder \texttt{lib/}. Then, one may include these libraries to the Java classpath either by extending
the enironment variable, e.g. via \texttt{YADA, YADA, YADA} under BASH.

\section{Python}
For some applications, Python 2.4.3 or later.

\chapter{Applications}
\section{GSRf}
To be announced later.

%\begin{thebibliography}{9}
%\end{thebibliography}

\end{document}
